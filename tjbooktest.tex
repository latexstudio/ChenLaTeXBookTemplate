%!Mode:: "TeX:Hard:UTF-8"
%!TEX program = xelatex
\documentclass{tjbook}
\title{一个中文书籍的\LaTeX{}模板}
\DedicatedTo{需要一个书籍模板的人}
\subtitle{这里可以放一个副标题}
\author{张三}% 默认值
\BookSeries{计算机应用技术丛书}% 可取消
\BookIntroduction{这个图书模板是在群主网站上的一个封面模板的基础上改写而成的,设定了一些图书出版要素,设计了封面、扉页及版权页和封底的样式,修改了chapter的样式,并提供了几个选项可切换色彩风格,其余则维持book基本文档类的设定。\par 图书模板部分代码的完成得到了林莲枝大神的帮助,在此表示感谢。由于作者水平有限,模板代码编写不恰当之处还请用户提出批评和指正。\par 感谢造字工坊提供了刻宋、郎宋和黄金时代三种非商业可免费下载使用的字体。\par 感谢谷歌提供自由使用的思源宋体、思源黑体。\par 感谢文泉驿提供的开源的文泉驿等宽微米黑字体。}
\Publisher{\LaTeX Studio 出版社} %默认值
%\Designer{张三丰}
%\Editor{张三丰}
\edition{1}% 可取消
\Price{26.5}
\isbn{978-80-7340-097-2}

\begin{document}
\maketitle
\makeflypage
\frontmatter
\tableofcontents

\mainmatter
\chapter{说明}
首先我们简要说明一下目前已经开发出来的模板功能,包括字体选择和颜色选择。
\section{颜色选择}
目前颜色选择共三种,green,orange和violet,默认为green。
\section{字体选择}
目前字体选择共两种,customfont和systemfont,默认为customfont。
若要使用customfont,需要安装相应的字体。
若要使用systemfont,则模板直接调用系统内部字体。

\section{几个可有可无的 tcolorbox 设置}

现有 \texttt{appledark}, \texttt{applelight}, \texttt{win10dark}, \texttt{win10light} 四个选项。
如果是供荧幕阅读的还好;如果是要打印出来的,除非您就是打印店的老板,不然还是不要多用 \texttt{*dark} 选项。

\begin{tcolorbox}[appledark,title={世界你好!}]
天气好吗?吃饱了吗?
\end{tcolorbox}

\begin{tcolorbox}[applelight,title={世界你好!}]
天气好吗?吃饱了吗?
\end{tcolorbox}

\begin{tcolorbox}[win10dark,title={世界你好!}]
天气好吗?吃饱了吗?
\end{tcolorbox}

\begin{tcolorbox}[win10light,title={世界你好!}]
天气好吗?吃饱了吗?
\end{tcolorbox}

\makebackcover
\end{document}
